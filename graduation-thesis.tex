\documentclass[a4paper, 11pt]{jsarticle}
\usepackage{lastpage}
\usepackage{setspace}
\usepackage{paralist}
\usepackage{enumitem}
\usepackage{amsmath,amssymb}
\usepackage{mathrsfs}
\makeatletter

\def\@maketitle{
\begin{flushright}
総ページ数:\pageref{LastPage}ページ
\end{flushright}
\hspace{0pt}
\vfill
\begin{flushleft}
\begin{spacing}{1.3}
{\large {2020年度 卒業研究}}
  \item[{\large{主査}}]  {\large{高橋 利枝 先生}}
  \leftskip = 3zw
  \item[{\large{題目}}]  {\large{メカニズムデザインを用いた、心理的安全性を向上させる組織内の\\{\hspace{42pt}}制度デザインの提案}}
  \leftskip = 3zw
  \vspace{2zh}
  \item[{\large{文化構想学部}}]  {\large{複合文化論系}}
  \leftskip = 3zw
  \item[{\large{学籍番号 }}]  {\large{1T160030-2}}
  \leftskip = 3zw
  \item[{\large{氏名   }}]  {\large{荒 翔太}}
\leftskip = 3zw
\end{spacing}
\end{flushleft}
\vfill
\hspace{0pt}
\par\vskip 1.5em
\thispagestyle{empty}
\clearpage
\addtocounter{page}{-2}
}
\makeatother

\begin{document}
\maketitle

\hspace{0pt}
\vfill
\tableofcontents
\thispagestyle{empty}
\vfill
\hspace{0pt}
\newpage
\section*{概要}
\section{序論}
我々の日々の活動はグループやチームでの活動といった形を取ることが多い.しかしそういった実態に反して,チームでの活動の効果性には常に不確実性が付き纏う.より個人でのパフォーマンスの高い人材をあるチームに多く所属させたからと言って,そのチームの効果性が必ず向上するとも言い切れず,反対にメンバー個々人のレベルからは想像もつかないパフォーマンスを披露するチームも存在する.チームの能力や効果性が単にメンバーのそれの総和とならないならば,如何にしてこれを向上させることができるのだろうか.

本稿ではチームの効果性に影響を与える因子として挙げられる心理的安全性に注目して,人的資源や機会といった不確定な要素を排除して,チームの効果性を向上させる手法について議論する.特に,どのチームにも適応させうる要素としてチーム内での制度に注目し,メカニズムデザインにおける耐戦略性を満たす制度設計を応用することで心理的安全性を担保するための制度設計について提案する.
\subsection{問題意識}
チームでの活動は,我々の日々の活動のうち多くの割合を占めているのにも関わらず,その効果性を向上させる一貫的な手法がそこに適用されることは多くない.適用されたとしてもそう言った手法の多くはメンバーの特性や能力に大きく依存し,一貫性があるものとは言い難い.また,メンバーの特性や能力に依存する手法はメンバー全員がその手法を把握していなければチームの効果性に寄与しないことが多い.

このような現状に対して,チームの効果性についての研究はこれまで多くなされてきた.しかし,その研究の多くはメンバーの性質や能力,マネージャーの能力,与えられる課題の性質,またはメンバーの心理的側面など依然として不確実性が高いものが多く,あらゆるチームに対して一貫性を持った解決策を提示しているとは言い難い.
\subsection{目的}
チームでの活動をモデル化し,人的影響やその他の不確実性の高い要因に依存せずにチームの効果性を高める手法を提案するところにある.

\section{先行研究}
チームでの活動の効果性を対象にした研究はこれまでにも数多くなされている.例えば,Stewart (2006) \cite{Stewart}はチームパフォーマンスに関する様々な相関を提供した.これによれば,チームのパフォーマンスとメンバー個々の能力と気質の間には正の相関があり,また,リーダーシップやチームに与えられたタスクなどの因子もこれと相関があると示した.特にタスクについては,パフォーマンスとの関係性は控えめであるとしながらも,そのタスクの種類によってメンバーの自主性や統制がどのようにパフォーマンスに影響するかが異なっていることを指摘した.

Mesmer-Magnus and Dechurch (2009) \cite{Mesmer}はチームのメンバーではなくチーム内での情報共有に注目し,情報の共有がチームの成果にとって重要な要素だと明らかにした上で,チームで必要となる情報がうまく共有されることは少ないと述べている.チームのパフォーマンスについて議論するに当たってメンバーの個々の性質以外の要素に注目した別の例がWoolley, Williams and Chabris, Christopher and Pentland, Alex and Hashmi, Nada and Malone, Thomas (2010) \cite{Woolley}である.これによれば,人々の知能とパフォーマンスに関係があることと同様に,グループにも知能が存在し,それによってパフォーマンスを説明することが可能である.また,グループの知能はメンバーの発言機会の均等性,メンバーの社会的感受性の平均,メンバー内の女性比率と正の相関があることも示された.このうち,メンバー内の女性比率がグループの知能とそう感があるのは,一般に女性の方が男性よりも社会的感受性が高いためと説明されている.この研究を受けGoogleが行った,効果的なチームの特徴を明らかにするプロジェクトにおいて,チームのメンバーが誰であるかよりもチームがどのように協力しているかこそがチームの効果性に影響を与えることが突き止められている (Google\cite{Google}) .チームに効果性に対する重要度が高い因子として,その重要度が高い順に
\begin{inparaenum}[1)]
  \item 心理的安全性
  \item 相互信頼
  \item 構造と明確さ
  \item 仕事の意味
  \item インパクト
\end{inparaenum}
の5つが挙げられた.
\subsection{心理的安全性}
ここで,先述のGoogle\cite{Google}においてチームの効果性に対して最も大きな影響を与える因子である心理的安全性に注目したい.心理的安全性とは,対人関係においてリスクのある行動を取ったときの,その結果に対する共通した認識のことを指し,より具体的には,「無知、無能、ネガティブ、邪魔だと思われる可能性のある行動をしても、このチームなら大丈夫だ」と信じられるかどうかを示す概念である.
心理的安全性という概念を最初に扱ったEdmondson (1999) \cite{Edmondson}は,心理的安全性がチームの効果性に対して影響を与えることを突き止める仮定で,チームの効果性とチームの学習行動との間に関係があることを明らかにした.チームでのある成果に対してそれを学習の機会と捉え,改善を続けていくチームこそがより良いチームの成果を形成することがここで示された.このような効果性と学習行動の関係性に対し,心理的安全性は学習行動に影響を与える.心理的安全性が確保されたチームはそうでないチームよりも,ある成果に対して学習行動を伴うことが示唆されており,心理的安全性は学習行動を媒体とすることで間接的にチームの成果に影響を与えることが示された.また,Edmondson (2004) \cite{Russell}はこれを補強し,チームにおいてメンバーとリーダーの間の信頼関係こそが,心理的安全性に最も寄与する要素である可能性が高いと述べた.

\section{方針}
本稿の目的は序論でも述べた通りチームの効果性を高めることである.心理的安全性を高めると学習行動を通してチームの効果性に間接的に良い影響を与えることができるので,不確実性の高い人的な要因や状況に依存せずこれを高めることによって,チームの効果性を高める結果を目指す.心理的安全性は,チームのメンバーの精神状況や特性に大きく関わるものであるが,Google\cite{GoogleDoc}はこれを高める個人的な取り組みをいくつか提案している.その中には「自分の意見を述べる」という要素が含まれており,これは全員がこのような積極的に自身の選好を表明することによって質問や学習行動の機会となり得る発言を促す効果があると考えられる.本稿では,不確実性の高い要因に依存しないチームの要素として,チーム内で運用されるルールや制度に対して操作を加え,チーム内での心理的安全性を高めることで間接的にチームの効果性を高める手法を検討する.

チームの制度について議論していくに当たって,ミクロ経済学やゲーム理論の応用分野であるメカニズムデザインの考え方を用いる.メカニズムデザインは元来,制度設計を通して投票方式や市場デザインをより良いものもしていくことを目指す分野である.この中に,耐戦略性という概念があり,これはその制度内の全員が自身の選好を正直に表明することが最善の選択肢となる制度の特徴を指す.これは言い換えれば虚偽の選好表明をする誘引をなくす制度の特徴のことであり,虚偽の選好への誘引をなくすことは即ち「自分の意見を述べる」理由を制度によって創出することに繋がり得る.ここでは選好を表明しないという行為も,選好が無差別であると表明するることと見なし,虚偽の選好表明に含む.チーム内での活動において,この耐戦略性を満たす制度を構築することが可能ならば,心理的安全性を仲介役とし,チームの効果性に制度が寄与することが可能になるのではないだろうか.ただ,一般に耐戦略性を満たすメカニズムは多くなく,満たしたとしても多くの条件下では独裁制に行き着くことが知られている.しかし,ゲーム理論における混合戦略均衡の場合と,準線形環境下では単純な独裁制では無いメカニズムによって耐戦略性を満たすことができるので,チームの活動をこれに当てはめることを考える.また,これを直接応用することが難しい局面には,投票ルールなどに関する研究分野である社会的選択理論の考え方を一部用いることを検討する.

\subsection{メカニズムデザイン}
メカニズムデザインとは投票方式や市場デザインの制度を設計することにより,より良い意思決定を可能にしようという分野である.メカニズムデザインの考え方の中では,意思決定の場面をモデル化し各要素間の関係性を表すことによって,どのような制度が最適かを探っていく.その中で個人の集合を\(I = \{1,2,\dots,n\}\)と表し,その組織のある意思決定の帰結の集合を\(X\)で表す.各個人\(i \in I\)は\(X\)上に選好\(\succsim_i\)を持つ.個人\(i\)が取りうる選好全てからなる集合を\(\mathscr{D}_i\)で表し,その全員分の組を
\(\mathscr{D} \equiv (\mathscr{D}_i)_{i \in I}\)により表し,これを評価関数プロファイルと呼ぶ.また,その全員分の直積
\[\mathscr{D}_I \equiv \mathscr{D}_1 \times \mathscr{D}_2 \times \cdots \times \mathscr{D}_n\]
をドメインと呼ぶ.このとき,メカニズムデザインの操作の対象となるのが,この\(\mathscr{D}\)と,ドメインからある帰結への関数である\(f: \mathscr{D}_I \rightarrow X\)とから成る組\[(\mathscr{D}, f)\]である.この\(f\)を社会的選択関数と呼び,この\((\mathscr{D}, f)\)をメカニズムと呼ぶ.
\subsection{社会的選択理論}
社会的選択理論とは,hoge

\section{チームに対するメカニズムデザイン的手法の制度的適用}
チームに対してメカニズムを通じて耐戦略性を付与するにあたり,耐戦略性以外にも効率性を満たすよう考える必要がある.ここでの効率性とは,パレート効率的,つまりどの個人にも損をさせることなくある個人に徳をさせることが無い状態を満たす配分の性質を指す.効率性を満たさない制度においては,メンバーにとっては例えばいかなる意見を述べても無駄であるが故に,虚偽の選好を表明する誘引が無いといった誤った方法を採用してしまう危険性を孕んでいるので効率性の確保は必要である.効率性と耐戦略性を満たすメカニズムは,その意思決定の環境によって二つ存在する.一つは確率的環境における意思決定で,この環境下では,意思決定の結果が混合戦略として表現される.つまり,この環境下では選び取られる帰結はいくつかの選択肢の確率の中から導き出されるので,全く同じ条件下で複数回意思決定したとしても,毎回同じ帰結を導くとは限らない.もうもう一方が準線形環境における意思決定である.この環境の下では,もし完全に同じ条件下で意思決定を行ったならば,その結果は必ず同じものになる.また,準線形環境には人々の選好への仮定も含まれているが,本稿ではチームでの意思決定をこちらの環境下のものとして扱っていく.準線形環境下では,グローヴスメカニズムというメカニズムをチーム内での制度として採用することで,チームとしての意思決定において効率性と耐戦略性を満たすことができる.しかし,グローヴスメカニズムをチームでの意思決定に用いるにはいくつかの障壁がある.
まず,グローヴスメカニズムはそもそも,社会的選択関数\(f\)が効率性を満たす前提のもとで成り立っているメカニズムである.つまり,これを用いるにはメンバーからの各選択肢への評価から,効率的な選択肢を選択,実行できる仕組みを設定する必要がある.
さらに,グローヴスメカニズムを含むあらゆるメカニズムは選択肢の集合が所与であることを前提としているが,チームでの意思決定においては選択肢を創出することからがそのチームの役割になることも多い.選択肢を創出する段階のことを考慮せずにグローヴスメカニズムの考え方を適用させると,かえってメンバーの真の選好を歪ませてしまうことも考えられる.

\subsection{報酬系の適用}
グローヴスメカニズムの考え方を適当に応用するために,チームでの意思決定過程をモデルに落とし込む必要がある.まず,これを適用するためにチームでの意思決定環境を準線形環境に落とし込む必要がある.なぜなら,グローヴスメカニズムは準線形環境を前提としているメカニズムだからだ.準線形環境においては,人々は社会的選択肢と金銭について準線形な選好を持つ.そもそもグローヴスメカニズムが準線形環境を前提としているのは,各選択肢へのある個人の効用を,その選択肢への金銭換算価値とその選択によって起こる彼への金銭転移額の和として表現するのにその仮定が必要だからだ.すなわち,社会的選択肢への評価と金銭価値を互換性を持って扱うためにこの仮定を置いているのだ.この仮定は,グローヴスメカニズムにおいて,選択肢の決定後に各個人が得られる報酬がその選択肢に対する評価値 (金銭換算価値) によって決定されるという形でメカニズムの報酬系に現れる.

グローヴスメカニズムはその報酬系によって耐戦略性を満たすことを試みるメカニズムである.このメカニズムでは,真の選好を表明することに対してインセンティブを与えるのではなく,虚偽の選好を表明することに対してインセンティブが生まれないように報酬が設定される.真の選好表明に対してインセンティブに対してインセンティブを与えようにも,表明されたある選好が真のものであるか虚偽のものであるかはその選好を表明した本人にしか分からず,結果として虚偽の選好表明に値してその選好が真だと言うだけでそのインセンティブを受け取れてしまうだ.そこでグローヴスメカニズムでは虚偽の選好を表明する誘引を奪うアプローチが採用されている.グローヴスメカニズムにおける個人\(i\)への報酬は,任意の\(v \in \mathscr{D}_I\)について
\[\sum_{j \neq i}v_j(d(v)) + h_i(v_{-i})\]
で表される.このとき,\(d: \mathscr{D}_I \rightarrow A\)を個人の選好集合の直積から実際に選びとる選択肢への関数,\(v_i : A \rightarrow \mathbb{R}\)を,\(i\)のある選択肢集合\(A\)からその評価を表す実数\(\mathbb{R}\)への関数,\(h_i(v_{-i})\)を任意の定数とする.なお,この報酬を導出するにおいて\(d\)は効率性を満たす仮定が置かれている.
つまり,グローヴスメカニズムの適用された条件下において,個人\(i\)が得ることができる報酬は,\(i\)自身以外の全ての個人の,最終的に選択された選択肢への評価の総和に依存する.この条件下では,与えられる報酬が,表明した自身の選好に全く依存しないため,虚偽の選好を表明する誘引がなくなる.なお,ただ単にこの報酬が設定されたからと言って,この報酬から制度が耐戦略性を満たすことを理解できる人は多くないだろう.Masuda, Sakai, Serizawa and Wakayama (2019) \cite{Sakai2019}によると,耐戦略性を満たすメカニズムの下においても,自身の選好を正直に表明しない人々は少なくないが,耐戦略性の説明をしっかりと与えることによって,そうした人の割合を下げることができる.これはこのメカニズムを適用する際にも重要な示唆となるだろう.

実際にグローヴスメカニズムを定数\(h\)を常に\(0\)を返す関数に設定してシンプルに適用してしまうと,収入源が無く\(I\)に対して予算の問題から報酬が支払えなくなってしまうので,これを調整したメカニズムが用いられることが現実では多い.こうしたメカニズムではまず最初に全ての個人が幾らかの金銭を支払い,それを再分配する形をとることが多い.実際にチームにこのメカニズムを適用することを考えたとき,これを実行できるチームであれば問題ないのだが,意思決定の度にチーム内で金銭の移動が起こることは好ましくない場合も多くあるのではないだろうか.そこで,これに代るものを考えたい.グローヴスメカニズムで仮定される準線形環境の中心となる考え方は,社会的選択肢への選好と報酬 (金銭) への選好との間に互換性を持たせることであり,これを満たしているものであれば代替可能であると考える.報酬そのものを金銭からずらすことなく,チームの内部での金銭転移をなくす方法として,報酬をチーム外からの賞与といった形で与えるといった方法が考えられる.しかし,ここではそもそも報酬を金銭以外のもので置き換えることを考えたい.これには理由が二つある.一つ目は,そもそもチームに対して賞与などを支払う金銭的余裕がないチームも存在するという理由だ.学生のチームや,営利を目的としていないチームにとってはそもそも金銭をメンバーに対して支払うことは難しい,または好ましくないと判断されることも少なくないのではないだろうか.もう一点は金銭を報酬として与えることがチームにとって悪影響を及ぼしかねないといった理由が挙げられる.Deci (1971) \cite{Deci}は金銭が外部報酬として用いられた時に内的動機が低下することを明らかにしている.チームの活動において,メンバーの内的動機が低下することがチームの効果性に対して致命的な影響を与えかねないことは自明であり,チームの効果性を高めるための制度がそれを低下させる帰結を招くようでは元も子もない.この問題を避けるためにも,チームでの活動に際してはその報酬を金銭ではないもので代替する必要がある.Deci (1972) \cite{Deci2}が明らかにしたところによると,金銭が内的動機を弱めるのに対して,社会的な承認はこれを高める.つまり,社会的選択肢と互換可能な形で社会的承認を配分することが可能であれば,メンバーの動機に良い影響を与えた上でグローヴスメカニズムを適用することができる.承認を配分する方法として,ここではメンバー各人への評価を配分されるべき報酬に従って高める手法を提案したい.グローヴスメカニズムで帰結として配分される報酬の量は各人によって異なる上,社会的選択肢への評価と互換性を持ってなくてはならない以上,人によっての差異を与えづらい,ただ賞賛を与えるなどの形よりも社会的選択肢への評価と互換できるような評価制度における評価を報酬として与える方がより適当と考える.元来のグローヴスメカニズムにおいては,社会的選択肢への評価はその金銭換算価値によって表現されていたが,ここでは社会的選択理論におけるスコアリングルールを採用し,ある選択肢に与えられたスコアをそのままその評価とする.
\subsection{効率性の担保}
グローヴスメカニズムは効率性と対戦略性を満たすメカニズムであり,そのうち対戦略性については配分によってこれを満たすよう設計してある.しかし,こと効率性についてはグローヴスメカニズムの証明においても,決定関数\(d\)が効率性を満たすよう仮定を置いており,これを満たす方法についての示唆は与えられていない.加えて,メカニズムの証明がそもそも効率性の仮定の上に成り立っているならば,如何に配分を調整したとしても意思決定の効率性を満たすことなくして対戦略性を満たすメカニズムを構築することはできない.そこで,グローヴスメカニズムを実用するにあたり意思決定の効率性を担保する必要がある.この「効率性」がパレート効率的である性質を表すものであることは前述の通りである.ただ,準線形環境の下でのみ,決定関数\(d\)が効率性を満たすことと,
\begin{equation}
\label{efficiency}
\sum_{i \in I}v_i(d(v)) = \max_{a \in A}\sum_{i \in I}v_i(a) \quad \forall v \in \mathscr{D}_I
\end{equation}
を満たすことは同値である.即ち,\(d\)によって決定されたある一つの選択肢に対する,全員の評価の総和が,他のどの選択肢のそれよりも高い場合,その決定関数\(d\)は効率性を満たす.これを満たす\(d\)を設定するために,社会的選択理論におけるスコアリングルールを用いる.社会的選択理論は投票ルールなど,複数の選択肢から一つの選択肢を複数人で選びとる手法に関する分野である.スコアリングルールとは,各選択肢に対して各個人が点数付けを行い,その結果を集計することで最終的に選ばれる選択肢を決定するルールを指す.我々がよく知る手法である多数決もこの一種であり,最も気に入った選択肢に1を,そうでない選択肢全てに0の点数付けを行い,その結果集めた点数の総和が最も大きい選択肢に決定する,スコアリングルールのやや極端な一例である.ここでスコアリングルールの採用を検討している理由は大きく分けて二つある.

一つ目はグローヴスメカニズムの性質によるものだ.まず,グローヴスメカニズムはその報酬の形を見ればわかる通り,各人の選択肢への評価が定量的に比較可能で,かつ相対的な評価ではなく絶対的な評価でなければならない.なぜなら,グローヴスメカニズムにおける個人\(i\)への報酬は\(i\)以外の全員の,採用された選択肢への評価の総和となるからだ.つまり,選択肢に対して与えられる評価は,加算可能でかつ定量的に表現することが可能である必要がある.その意味において,スコアリングルールでは文字通り各選択肢への評価を点数,それも一般的なルールにおいては自然数で表すのでこの両点を満たす.また,この点数付けに自然数を用いる点も,今回のチーム内での意思決定という条件に非常に相性が良い.通常のグローヴスメカニズムにおいては,選択肢に負の評価を与えることが可能である.これは,予算制約のある環境下においては,財源がなくなり実行可能性がなくなってしまうのに貢献するが,今回のルールでは予算の点から見た実行可能性は考慮しないのでこれを正の評価に限定することに問題はない.むしろ,ある意思決定の結果として報酬が負の数となってしまった場合,意思決定に貢献したはずのメンバーに対する報酬が負の数,つまり彼らへの評価を下げることに繋がりかねない.これに対し,選ばれる選択肢は常に得点が最大であるのでこれが負の数に成ることはないとする反論が考えられるが,例えば好ましくない選択肢の中から最も損失が少ないであろう選択肢を選ぶ意思決定に際したときに,スコアリングに負の点数付けを許すと,与えられる点数が全て負の数になり,得点数が最大の選択肢でも,その得点の総和が負の数となる場合があることは自明だろう.

二つ目は合理性,効率性に関するものだ.今回対象にしているような複数選択肢,複数人での意思決定の際に用いる決定方略は概ね多数決であろう.これは何も明示的に多数決である場合のみを指すのではない.多くの場合,選択肢についての話し合いは行われるだろう.しかし,最終的に満場一致のように見える意思決定であっても,そこには多数決,つまり\((1,0)\)のスコアリングが存在する.話し合いによって全ての個人の意見が一致した後での\((1,0)\)のスコアリングが行われているのである.多数決に当てはまらない場合では,独裁制のような決定方略が用いられることもあるだろう.ある一人の鶴の一声で決まる場合などは独裁制に相当する.独裁制の下での決定が常に効率性を満たすものならば,耐戦略性と効率性を満たすことは事実可能である.しかしこれには問題が二点ある.一つが,チーム自体の効果性に関するものだ.独裁制はGoogle\cite{Google}の示した「心理的安全性」に止まらず,非独裁者の「仕事の意味」「相互信頼」を崩壊させ得る.本稿の最終的な目的がチームの効果性である以上,それに反する制度を採用することはできない.そしてもう一つが,そもそも独裁者が常に効率的な選択を行うことが非常に難しいという事実に関するものだ.

本稿ではボルダルールの採用を検討したい.ボルダルールはスコアリングルールの一つである.ボルダルールでは,選択肢が\(n\)個あったとき,それぞれの投票参加者は各選択肢に1位から\(n\)位までの順位を付け,1位に\(n\)点,2位に\(n-1\)点,\dots ,\(n\)位に1点を付与し,全参加者の与えた得点の総和をその選択肢の得点とし,それが最大の選択肢を最終的に選択する.このルールは一般にパレート効率性を満たす帰結を選ぶことを担保するものではなく,多数決の欠点である,ペア全敗者の選択,つまり誰もが望んでいない選択肢の採用を避けるルールであるが, (\ref{efficiency}) 式より,準線形環境下ではボルダルールで効率性を満たすことができる.

\subsection{選択肢の創出段階における工夫}
ここまではチームでの活動にグローヴスメカニズムを適用することが可能になるようにこれを調整してきた.しかし,グローヴスメカニズムの適用外範囲における,選択肢の創出段階においても操作を加える必要gある.ここまで述べたメカニズムを,選択肢の創出段階に操作を加えずにチームでの活動に適用させてしまうと,メンバーの選好を歪ませかねない.グローヴスメカニズムを含むメカニズムデザインでは,選択肢を所与として考えているが,実際のチームでの活動では選択肢集合\(A\)を考える段階からも活動に含まれることが多いだろう.先に述べた通り,メンバーに選好を正直に表明してもらうためには,グローヴスメカニズムで最終的に与えられる報酬も含めてこのメカニズムと,それが耐戦略性を満たしていることについて説明を与える必要がある.先にメンバーにこのメカニズムについて説明を与えることは,即ちこの制度の下で自身に与えられる利益についての情報をメンバーに与えることだ.このメカニズムにおいては自分以外のメンバーの選択肢への評価が自身の報酬に応じて決定されるので,選択肢を考える段階で自分の真の選好を隠し,自分以外のメンバーが好むような選択肢を提案する誘引がメンバーに生まれる.これを防ぐためには,自分以外のメンバーの選好を知らないうちに選択肢の提案を行う手法が考えられる.通常のチームでの活動であれば,チームのメンバーで集まって選択肢を考えることが多いだろうが,このときにはある個人が提案した選択肢の内容や傾向から彼の選好をある程度窺い知ることが可能になってしまうであろう.そこで,他のメンバーの選好を知る前に選択肢の提案を行うことができれば,これをある程度解決することができる.Mullen, Johnson and Salas (1991) \cite{Mullen}は,個人でアイデアを考えることの生産性はブレインストーミングでのそれに劣らず,むしろ通常のブレインストーミングによる生産性を大幅に上回ることを突き止めている.つまり,ここで戦略的な操作を防ぐために選択肢の提案を個々人に任せたとしても,そのことによる生産性の低下を心配する必要はない.

\subsection{適用範囲の調整}


\section{結論}
\section{研究における課題と限界}
\section*{参考文献}
\subsection*{引用文献}
\begingroup
\renewcommand{\section}[2]{}
\bibliography{references}
\bibliographystyle{junsrt}
\endgroup
\subsection*{参考文献}
\begin{itemize}[leftmargin=*]
\item[] 豊貴坂井. メカニズムデザイン : 資源配分制度の設計とインセンティブ / 坂井豊貴, 藤中裕二, 若山琢磨著. ミネルヴァ書房, 京都, 2008.8.
\end{itemize}
\end{document}
