\documentclass[a4paper, 11pt]{jsarticle}  
\usepackage{lastpage}
\usepackage{setspace}
\makeatletter

\def\@maketitle{
\begin{flushright}
総ページ数:\pageref{LastPage}ページ
\end{flushright}
\hspace{0pt}
\vfill
\begin{flushleft}
\begin{spacing}{1.3}
{\large {2020年度 卒業研究}}
  \item[{\large{主査}}]  {\large{高橋 利枝 先生}}
  \leftskip = 3zw
  \item[{\large{題目}}]  {\large{メカニズムデザインを用いた、心理的安全性を向上させる組織内の\\{\hspace{42pt}}制度デザインの提案}}
  \leftskip = 3zw
  \vspace{2zh}
  \item[{\large{文化構想学部}}]  {\large{複合文化論系}}
  \leftskip = 3zw
  \item[{\large{学籍番号 }}]  {\large{1T160030-2}}
  \leftskip = 3zw
  \item[{\large{氏名   }}]  {\large{荒 翔太}}
\leftskip = 3zw
\end{spacing}
\end{flushleft}
\vfill
\hspace{0pt}
\par\vskip 1.5em
\thispagestyle{empty}
\clearpage
\addtocounter{page}{-2}
}
\makeatother

\begin{document}
\maketitle

\hspace{0pt}
\vfill
\tableofcontents
\thispagestyle{empty}
\vfill
\hspace{0pt}
\newpage
\section*{概要}
あああああああああああああああああああああああああああああああああああああああああああああああああああああああああああああああああああああああああああああああああああああああああああああああああああああああああああああああああああああああああああああああああああああああああああああああああああああああああああああああああああああああああああああああああああああああああああああああああああああああああああああああああああああああああああああああああああああああああああああああああああああああああああああああああああああああああああああああああああああああああああああああああああああああああああああああああああああああああああああああああああああああああああああああああああああああああああああああああああああああああああああああああああああああああああああああああああああああああああああああああああああああああああああああああああああああああ
\section{序論}
我々の日々の活動はグループやチームでの活動といった形を取ることが多い.しかしそういった実態に反して,チームでの活動の効果性には常に不確実性が付き纏う.より個人でのパフォーマンスの高い人材をあるチームに多く所属させたからと言って,そのチームの効果性が必ず向上するとも言い切れず,反対にメンバー個々人のレベルからは想像もつかないパフォーマンスを披露するチームも存在する.チームの能力や効果性が単にメンバーのそれの総和とならないならば,如何にしてこれを向上させることができるのだろうか.

本稿ではチームの効果性に影響を与える因子として挙げられる心理的安全性に注目して,人的資源や機会といった不確定な要素を排除して,チームの効果性を向上させる手法について議論する.特に,どのチームにも適応させうる要素としてチーム内での制度に注目し,メカニズムデザインにおける耐戦略性を満たす制度設計を応用することで心理的安全性を担保するための制度設計について提案する.
\subsection{問題意識}
チームでの活動は,我々の日々の活動のうち多くの割合を占めているのにも関わらず,その効果性を向上させる一貫的な手法がそこに適用されることは多くない.適用されたとしてもそう言った手法の多くはメンバーの特性や能力に大きく依存し,一貫性があるものとは言い難い.また,メンバーの特性や能力に依存する手法はメンバー全員がその手法を把握していなければチームの効果性に寄与しないことが多い.

このような現状に対して,チームの効果性についての研究はこれまで多くなされてきた.しかし,その研究の多くはメンバーの性質や能力,マネージャーの能力,与えられる課題の性質,またはメンバーの心理的側面など依然として不確実性が高いものが多く,あらゆるチームに対して一貫性を持った解決策を提示しているとは言い難い.
\subsection{目的}
チームでの活動をモデル化し,人的影響やその他の不確実性の高い要因に依存せずにチームの効果性を高める手法を提案するところにある.

\subsection{方針}
本稿では,不確実性の高い要因に依存しないチームの要素としてチーム内で運用されるルールや制度に対して操作を加えることを検討する.また,Googleの研究でチームの効果性に最も影響を与える因子として挙げられた心理的安全性に注目し.チーム内での制度によって心理的安全性を高める手法について考察,提案していく.チーム内の制度については,ミクロ経済学やゲーム理論の応用分野であり,より良い意思決定を制度を用いて促すことを目標とする,メカニズムデザインの考え方を応用する.このメカニズムデザインのモデルをチーム内での意思決定に直接応用するのが難しい局面には,投票ルールなどの研究分野である社会的選択理論の考え方を用いる.

\section{先行研究}
\subsection{チームパフォーマンスについての先行研究}
ああああああああああああああああああああああああああああああああああああああああああああああああああああああ あああああああああああああああああああああああああああああああああああああああああああああああああああああああああああああああああああああああああああああああああああああああああああああああああああああああああああああああああああああああああああああああああああああああああああああああああああああああああああああああああああああああああああああああああああああああああああああああああああああああああああああああああああああああああああああああああああああああああああああああああああああああああああああああああああああああああああああああああああああああああああああああああああああああああああああああああああああああああああああああああああああああああああああああああああああああああああああああああああああああああああああああああああ
\subsubsection{心理的安全性}
心理的安全性という概念を最初に扱ったAmy Edmondson\cite{amy_edmondson}は、心理的安全性がいかにしてチームの効果性に影響を与えるかを明らかにした。
\subsection{メカニズムデザイン}
あああああああああああああああああああああああああああああああああああああああああああああああああああああああああああああああああああああああああああああああああああああああああああああああああああああああああああああああああああああああああああああああああああああああああああああああああああああああああああああああああああああああああああああああああああああああああああああああああああああああああああああああああああああああああああああああああああああああああああああああああああああああああああああああああああああああああああああああああああああああああああああああああああああああああああああああああああああああああああああああああああああああああああああああああああああああああああああああああああああああああああああああああああああああああああああああああああああああああああああああああああああああああああああああああああああああああ
\subsection{社会的選択理論}
あああああああああああああああああああああああああああああああああああああああああああああああああああああああああああああああああああああああああああああああああああああああああああああああああああああああああああああああああああああああああああああああああああああああああああああああああああああああああああああああああああああああああああああああああああああああああああああああああああああああああああああああああああああああああああああああああああああああああああああああああああああああああああああああああああああああああああああああああああああああああああああああああああああああああああああああああああああああああああああああああああああああああああああああああああああああああああああああああああああああああああああああああああああああああああああああああああああああああああああああああああああああああああああああああああああああああ
\section{チームに対するメカニズムデザイン的手法の制度的適用}
効率性と耐戦略性の両方を満たすメカニズムはその環境が確立的環境であるか準線形環境であるかによって異なるが,チームとしての意思決定は確立的に行われるとは考えづらいので,今回は準線形環境として環境下でのメカニズムを用いる.準線形環境下では,グローヴスメカニズムをチーム内での制度として採用することで,チームとしての意思決定において効率性と耐戦略性を満たすことができる.しかし,グローヴスメカニズムをチームでの意思決定に用いるにはいくつかの障壁がある.
まず,グローヴスメカニズムはそもそも,意思決定における決定関数\(d: V \rightarrow A\)が効率性を満たす前提のもとで成り立っているメカニズムである.つまり,これを用いるにはメンバーからの各選択肢への評価から,効率的な選択肢を選択,実行できる仕組みを設定する必要がある.
さらに,グローヴスメカニズムを含むあらゆるメカニズムは選択肢集合\(A\)が所与であることを前提としているが,チームでの意思決定においては選択肢を創出することからがそのチームの役割になることも多い.選択肢を創出する段階のことを考慮せずにグローヴスメカニズムの考え方を適用させると,かえってメンバーの真の選好を歪ませてしまうことも考えられる.
\subsection{環境のモデル化}
グローヴスメカニズムの考え方を適当に応用するために,チームでの意思決定過程をモデルに落とし込む必要がある.まず,これを適用するためにチームでの意思決定環境を準線形環境として定義する必要がある.準線形環境とは,人々が社会的選択肢と金銭について準線形な選好を持つ環境を指す.hoge

\subsection{グローヴスメカニズムの適用}
あああああああああああああああああああああああああああああああああああああああああああああああああああああああああああああああああああああああああああああああああああああああああああああああああああああああああああああああああああああああああああああああああああああああああああああああああああああああああああああああああああああああああああああああああああああああああああああああああああああああああああああああああああああああああああああああああああああああああああああああああああああああああああああああああああああああああああああああああああああああああああああああああああああああああああああああああああああああああああああああああああああああああああああああああああああああああああああああああああああああああああああああああああああああああああああああああああああああああああああああああああああああああああああああああああああああああ
\subsubsection{効率性の担保}
あああああああああああああああああああああああああああああああああああああああああああああああああああああああああああああああああああああああああああああああああああああああああああああああああああああああああああああああああああああああああああああああああああああああああああああああああああああああああああああああああああああああああああああああああああああああああああああああああああああああああああああああああああああああああああああああああああああああああああああああああああああああああああああああああああああああああああああああああああああああああああああああああああああああああああああああああああああああああああああああああああああああああああああああああああああああああああああああああああああああああああああああああああああああああああああああああああああああああああああああああああああああああああああああああああああああああ
\subsubsection{選択肢の創出段階における工夫}
あああああああああああああああああああああああああああああああああああああああああああああああああああああああああああああああああああああああああああああああああああああああああああああああああああああああああああああああああああああああああああああああああああああああああああああああああああああああああああああああああああああああああああああああああああああああああああああああああああああああああああああああああああああああああああああああああああああああああああああああああああああああああああああああああああああああああああああああああああああああああああああああああああああああああああああああああああああああああああああああああああああああああああああああああああああああああああああああああああああああああああああああああああああああああああああああああああああああああああああああああああああああああああああああああああああああああ
\subsubsection{適用範囲の調整}
あああああああああああああああああああああああああああああああああああああああああああああああああああああああああああああああああああああああああああああああああああああああああああああああああああああああああああああああああああああああああああああああああああああああああああああああああああああああああああああああああああああああああああああああああああああああああああああああああああああああああああああああああああああああああああああああああああああああああああああああああああああああああああああああああああああああああああああああああああああああああああああああああああああああああああああああああああああああああああああああああああああああああああああああああああああああああああああああああああああああああああああああああああああああああああああああああああああああああああああああああああああああああああああああああああああああああ
\section{結論}
あああああああああああああああああああああああああああああああああああああああああああああああああああああああああああああああああああああああああああああああああああああああああああああああああああああああああああああああああああああああああああああああああああああああああああああああああああああああああああああああああああああああああああああああああああああああああああああああああああああああああああああああああああああああああああああああああああああああああああああああああああああああああああああああああああああああああああああああああああああああああああああああああああああああああああああああああああああああああああああああああああああああああああああああああああああああああああああああああああああああああああああああああああああああああああああああああああああああああああああああああああああああああああああああああああああああああ
\section{研究における課題と限界}
あああああああああああああああああああああああああああああああああああああああああああああああああああああああああああああああああああああああああああああああああああああああああああああああああああああああああああああああああああああああああああああああああああああああああああああああああああああああああああああああああああああああああああああああああああああああああああああああああああああああああああああああああああああああああああああああああああああああああああああああああああああああああああああああああああああああああああああああああああああああああああああああああああああああああああああああああああああああああああああああああああああああああああああああああああああああああああああああああああああああああああああああああああああああああああああああああああああああああああああああああああああああああああああああああああああああああ
\section{参考文献リスト}
\begingroup
\renewcommand{\section}[2]{}
\begin{thebibliography}{99}
  \bibitem{amy_edmondson} Amy Edmondson, “Psychological Safety and Learning Behavior in Work Teams”, Administrative Science Quarterly, Vol. 44, No. 2, pp. 350-383, 1999
\end{thebibliography}
\endgroup
\section{付録}
\end{document}
