\documentclass[a4paper, 11pt]{jsarticle}  
\usepackage{lastpage}
\usepackage{setspace}
\usepackage{enumitem}
\makeatletter

\def\@maketitle{
\begin{flushright}
総ページ数:\pageref{LastPage}ページ
\end{flushright}
\hspace{0pt}
\vfill
\begin{flushleft}
\begin{spacing}{1.3}
{\large {2020年度 卒業研究}}
  \item[{\large{主査}}]  {\large{高橋 利枝 先生}}
  \leftskip = 3zw
  \item[{\large{題目}}]  {\large{メカニズムデザインを用いた、心理的安全性を向上させる組織内の\\{\hspace{42pt}}制度デザインの提案}}
  \leftskip = 3zw
  \vspace{2zh}
  \item[{\large{文化構想学部}}]  {\large{複合文化論系}}
  \leftskip = 3zw
  \item[{\large{学籍番号 }}]  {\large{1T160030-2}}
  \leftskip = 3zw
  \item[{\large{氏名   }}]  {\large{荒 翔太}}
\leftskip = 3zw
\end{spacing}
\end{flushleft}
\vfill
\hspace{0pt}
\par\vskip 1.5em
\thispagestyle{empty}
\clearpage
\addtocounter{page}{-2}
}
\makeatother

\begin{document}
\maketitle

\hspace{0pt}
\vfill
\tableofcontents
\thispagestyle{empty}
\vfill
\hspace{0pt}
\newpage
\section*{概要}
あああああああああああああああああああああああああああああああああああああああああああああああああああああああああああああああああああああああああああああああああああああああああああああああああああああああああああああああああああああああああああああああああああああああああああああああああああああああああああああああああああああああああああああああああああああああああああああああああああああああああああああああああああああああああああああああああああああああああああああああああああああああああああああああああああああああああああああああああああああああああああああああああああああああああああああああああああああああああああああああああああああああああああああああああああああああああああああああああああああああああああああああああああああああああああああああああああああああああああああああああああああああああああああああああああああああああ
\section{序論}
我々の日々の活動はグループやチームでの活動といった形を取ることが多い.しかしそういった実態に反して,チームでの活動の効果性には常に不確実性が付き纏う.より個人でのパフォーマンスの高い人材をあるチームに多く所属させたからと言って,そのチームの効果性が必ず向上するとも言い切れず,反対にメンバー個々人のレベルからは想像もつかないパフォーマンスを披露するチームも存在する.チームの能力や効果性が単にメンバーのそれの総和とならないならば,如何にしてこれを向上させることができるのだろうか.

本稿ではチームの効果性に影響を与える因子として挙げられる心理的安全性に注目して,人的資源や機会といった不確定な要素を排除して,チームの効果性を向上させる手法について議論する.特に,どのチームにも適応させうる要素としてチーム内での制度に注目し,メカニズムデザインにおける耐戦略性を満たす制度設計を応用することで心理的安全性を担保するための制度設計について提案する.
\subsection{問題意識}
チームでの活動は,我々の日々の活動のうち多くの割合を占めているのにも関わらず,その効果性を向上させる一貫的な手法がそこに適用されることは多くない.適用されたとしてもそう言った手法の多くはメンバーの特性や能力に大きく依存し,一貫性があるものとは言い難い.また,メンバーの特性や能力に依存する手法はメンバー全員がその手法を把握していなければチームの効果性に寄与しないことが多い.

このような現状に対して,チームの効果性についての研究はこれまで多くなされてきた.しかし,その研究の多くはメンバーの性質や能力,マネージャーの能力,与えられる課題の性質,またはメンバーの心理的側面など依然として不確実性が高いものが多く,あらゆるチームに対して一貫性を持った解決策を提示しているとは言い難い.
\subsection{目的}
チームでの活動をモデル化し,人的影響やその他の不確実性の高い要因に依存せずにチームの効果性を高める手法を提案するところにある.

\section{先行研究}
本稿では,不確実性の高い要因に依存しないチームの要素としてチーム内で運用されるルールや制度に対して操作を加えることを検討する.
チーム内での制度に対して操作を加えることで,
また,Googleの研究でチームの効果性に最も影響を与える因子として挙げられた心理的安全性に注目し.チーム内での制度によって心理的安全性を高める手法について考察,提案していく.チーム内の制度については,ミクロ経済学やゲーム理論の応用分野であり,より良い意思決定を制度を用いて促すことを目標とする,メカニズムデザインの考え方を応用する.このメカニズムデザインのモデルをチーム内での意思決定に直接応用するのが難しい局面には,投票ルールなどの研究分野である社会的選択理論の考え方を用いる.

\subsection{チームパフォーマンスについての先行研究}
チームでの活動の効果性を対象にした研究はこれまでにも数多くなされている.例えば,Stewart\cite{Stewart}はチームパフォーマンスに関する様々な相関を提供した.これによれば,チームのパフォーマンスとメンバー個々の能力と気質の間には正の相関があり,また,リーダーシップやチームに与えられたタスクなどの因子もこれと相関があると示した.特にタスクについては,パフォーマンスとの関係性は控えめであるとしながらも,そのタスクの種類によってメンバーの自主性や統制がどのようにパフォーマンスに影響するかが異なっていることを指摘した.

Mesmer-Magnus, Jessica and Dechurch, Leslie\cite{Mesmer}はチームのメンバーではなくチーム内での情報共有に注目し,情報の共有がチームの成果にとって重要な要素だと明らかにした上で,チームで必要となる情報がうまく共有されることは少ないと述べている.チームのパフォーマンスについて議論するに当たってメンバーの個々の性質以外の要素に注目した別の例がWoolley, Williams and Chabris, Christopher and Pentland, Alex and Hashmi, Nada and Malone, Thomas\cite{Woolley}である.これによれば,人々の知能とパフォーマンスに関係があることと同様に,グループにも知能が存在し,それによってパフォーマンスを説明することが可能である.また,グループの知能はメンバーの発言機会の均等性,メンバーの社会的感受性の平均,メンバー内の女性比率と正の相関があることも示された.このうち,メンバー内の女性比率がグループの知能とそう感があるのは,一般に女性の方が男性よりも社会t系感受性が高いためと説明されている.この研究を受けGoogleが行った,効果的なチームの特徴を明らかにするプロジェクトにおいて,チームのメンバーが誰であるかよりもチームがどのように協力しているかこそがチームの効果性に影響を与えることが突き止められている(Google\cite{Google}).具体的にはチームの効果性に対する重要度が高い順に,「心理的安全性」「相互信頼」「構造と明確さ」「仕事の意味」「インパクト」が因子として挙げられた.
\subsubsection{心理的安全性}
ここで,先述のGoogle\cite{Google}においてチームの効果性に対して最も大きな影響を与える因子である心理的安全性に注目したい.心理的安全性とは,対人関係においてリスクのある行動を取ったときの,その結果に対する共通した認識のことを指す.より具体的には,Google\cite{Google}の言葉を借りれば「無知、無能、ネガティブ、邪魔だと思われる可能性のある行動をしても、このチームなら大丈夫だ」と信じられるかどうかを示す概念である.
心理的安全性という概念を最初に扱ったEdmondson\cite{Edmondson}は,心理的安全性がチームの効果性に対して影響を与えることを突き止める仮定で,チームの効果性とチームの学習行動との間に関係があることを明らかにした.チームでのある成果に対してそれを学習の機会と捉え,改善を続けていくチームこそがより良いチームの成果を形成することがここで示された.このような効果性と学習行動の関係性に対し,心理的安全性は学習行動に影響を与える.心理的安全性が確保されたチームはそうでないチームよりも,ある成果に対して学習行動を伴うことが示唆されており,心理的安全性は学習行動を媒体とすることで間接的にチームの成果に影響を与えることが示された.

\section{方針}
本稿の目的は序論でも述べた通りチームの効果性を高めることである.心理的安全性を高めると学習行動を通してチームの効果性に間接的に良い影響を与えることができるので,不確実性の高い人的な要因や状況に依存せずこれを高めることによって,チームの効果性を高める結果を目指す.心理的安全性は,チームのメンバーの精神状況や特性に大きく関わるものであるが,Google\cite{GoogleDoc}はこれを高める個人的な取り組みをいくつか提案している.その中には「自分の意見を述べる」という要素が含まれており,これは全員がこのような積極的に自身の選好を表明することによって質問や学習行動の機会となり得る発言を促す効果があると考えられる.あらゆるチームに対して適応可能な一貫した要素として,チームの制度によってこれを実現していくことを本稿では検討していく.

メカニズムデザインの考え方を用いて,まさに自分の選好を表明することが最良の選択肢となる条件である,耐戦略性を満たした制度を構築することができれば,仕組みを通じて心理的安全性の担保に寄与できるだろう.一般に耐戦略性を満たすメカニズムは多くなく,多くの条件下では独裁制に行き着くことが知られている.しかし,ゲーム理論における混合戦略均衡の場合と,準線形環境下では単純な独裁制ではなくこれを満たすメカニズムが存在するので,これを当てはめることを検討する.


\subsection{メカニズムデザイン}
あああああああああああああああああああああああああああああああああああああああああああああああああああああああああああああああああああああああああああああああああああああああああああああああああああああああああああああああああああああああああああああああああああああああああああああああああああああああああああああああああああああああああああああああああああああああああああああああああああああああああああああああああああああああああああああああああああああああああああああああああああああああああああああああああああああああああああああああああああああああああああああああああああああああああああああああああああああああああああああああああああああああああああああああああああああああああああああああああああああああああああああああああああああああああああああああああああああああああああああああああああああああああああああああああああああああああ
\subsection{社会的選択理論}
あああああああああああああああああああああああああああああああああああああああああああああああああああああああああああああああああああああああああああああああああああああああああああああああああああああああああああああああああああああああああああああああああああああああああああああああああああああああああああああああああああああああああああああああああああああああああああああああああああああああああああああああああああああああああああああああああああああああああああああああああああああああああああああああああああああああああああああああああああああああああああああああああああああああああああああああああああああああああああああああああああああああああああああああああああああああああああああああああああああああああああああああああああああああああああああああああああああああああああああああああああああああああああああああああああああああああ

\section{チームに対するメカニズムデザイン的手法の制度的適用}
効率性と耐戦略性の両方を満たすメカニズムはその環境が確立的環境であるか準線形環境であるかによって異なるが,チームとしての意思決定は確立的に行われるとは考えづらいので,今回は準線形環境として環境下でのメカニズムを用いる.準線形環境下では,グローヴスメカニズムをチーム内での制度として採用することで,チームとしての意思決定において効率性と耐戦略性を満たすことができる.しかし,グローヴスメカニズムをチームでの意思決定に用いるにはいくつかの障壁がある.
まず,グローヴスメカニズムはそもそも,意思決定における決定関数,即ち各人の評価関数プロファイルから社会的選択肢への写像が効率性を満たす前提のもとで成り立っているメカニズムである.つまり,これを用いるにはメンバーからの各選択肢への評価から,効率的な選択肢を選択,実行できる仕組みを設定する必要がある.
さらに,グローヴスメカニズムを含むあらゆるメカニズムは選択肢の集合が所与であることを前提としているが,チームでの意思決定においては選択肢を創出することからがそのチームの役割になることも多い.選択肢を創出する段階のことを考慮せずにグローヴスメカニズムの考え方を適用させると,かえってメンバーの真の選好を歪ませてしまうことも考えられる.

\subsection{グローヴスメカニズムの適用}
グローヴスメカニズムの考え方を適当に応用するために,チームでの意思決定過程をモデルに落とし込む必要がある.まず,これを適用するためにチームでの意思決定環境を準線形環境として定義する必要がある.なぜなら,グローヴスメカニズムは,ある環境下では解が一つに定まる純粋戦略下で,尚且つ準線形環境を前提としているメカニズムだからだ.準線形環境とは,人々が社会的選択肢と金銭について準線形な選好を持つ環境を指す.そもそもグローヴスメカニズムが準線形環境を前提としているのは,各選択肢へのある個人の効用を,その選択肢への金銭換算価値とその選択によって起こる彼への金銭転移額の和として表現するのにその仮定が必要だからだ.すなわち,社会的選択肢への評価と金銭価値を互換性を持って扱うためにこの仮定を置いているのだ.この仮定は,グローヴスメカニズムにおいて,選択肢の決定後に各個人が得られる報酬がその選択肢に対する評価値(金銭換算価値)によって決定されるという形でメカニズムの報酬系に現れる.

グローヴスメカニズムをチームでの活動に適用するに当たってのチーム内での報酬系について考える.グローヴスメカニズムを純粋に適用するとすれば,予算制約の問題もあり一度チームの構成員から金銭を収集し,最後に報酬としてそれを再分配する形をとることになる.これを実行できるチームであれば問題ないのだが,実際に意思決定の度にチーム内で金銭の移動が起こることは好ましくない場合も多くあるのではないだろうか.そこで,これに代るものを考えたい.グローヴスメカニズムで仮定される準線形環境の中心となる考え方は,社会的選択肢への選好と報酬(金銭)への選好との間に互換性を持たせることであり,これを満たしているものであれば代替可能であると考える.報酬そのものを金銭からずらすことなく,チームの内部での金銭転移をなくす方法として,報酬をチーム外からの賞与といった形で与えるといった方法が考えられる.しかし,ここではそもそも報酬を金銭以外のもので置き換えることを考えたい.これには理由が二つある.一つ目は,そもそもチームに対して賞与などを支払う金銭的余裕がないチームも存在するという理由だ.学生のチームや,営利を目的としていないチームにとってはそもそも金銭をメンバーに対して支払うことは難しい,または好ましくないと判断されることも少なくないのではないだろうか.もう一点は金銭を報酬として与えることがチームにとって悪影響を及ぼしかねないといった理由が挙げられる.Deci(1971)\cite{Deci}は金銭が外部報酬として用いられた時に内的動機が低下することを明らかにしている.チームの活動において,メンバーの内的動機が低下することがチームの効果性に対して致命的な影響を与えかねないことは自明であり,チームの効果性を高めるための制度がそれを低下させる帰結を招くようでは元も子もない.この問題を避けるためにも,チームでの活動に際してはその報酬を金銭ではないもので代替する必要がある.Deci(1972)\cite{Deci2}が明らかにしたところによると,金銭が内的動機を弱めるのに対して,社会的な承認はこれを高める.つまり,社会的選択肢と互換可能な形で社会的承認を配分することが可能であれば,メンバーの動機に良い影響を与えた上でグローヴスメカニズムを適用することができる.承認を配分する方法として,ここではメンバー各人への評価を配分されるべき報酬に従って高める手法を提案したい.グローヴスメカニズムで帰結として配分される報酬の量は各人によって異なる上,社会的選択肢への評価と互換性を持ってなくてはならない以上,人によっての差異を与えづらい,ただ賞賛を与えるなどの形よりも社会的選択肢への評価と互換できるような評価制度における評価を報酬として与える方がより適当と考える.元来のグローヴスメカニズムにおいては,社会的選択肢への評価はその金銭換算価値によって表現されていたが,ここでは社会的選択理論におけるスコアリングルールを採用し,ある選択肢に与えられたスコアをそのままその評価とする.
\subsubsection{効率性の担保}
グローヴスメカニズムは効率性と対戦略性を満たすメカニズムであり,そのうち対戦略性については配分によってこれを満たすよう設計してある.しかし,こと効率性についてはグローヴスメカニズムの証明においても,ある決定関数が効率性を満たすよう仮定を置いており,これを満たす方法についての示唆は与えられていない.加えて,メカニズムの証明がそもそも効率性の仮定の上に成り立っているならば,如何に配分を調整したとしても意思決定の効率性を満たすことなくして対戦略性を満たすメカニズムを構築することはできない.そこで,グローヴスメカニズムを実用するにあたり意思決定の効率性を担保する必要がある.
あああああああああああああああああああああああああああああああああああああああああああああああああああああああああああああああああああああああああああああああああああああああああああああああああああああああああああああああああああああああああああああああああああああああああああああああああああああああああああああああああああああああああああああああああああああああああああああああああああああああああああああああああああああああああああああああああああああああああああああああああああああああああああああああああああああああああああああああああああああああああああああああああああああああああああああああああああああああああああああああああああああああああああああああああああああああああああああああああああああああああああああああああああああああああああああああああああああああああああああああああああああああああああああああああああああああああ
\subsubsection{選択肢の創出段階における工夫}
あああああああああああああああああああああああああああああああああああああああああああああああああああああああああああああああああああああああああああああああああああああああああああああああああああああああああああああああああああああああああああああああああああああああああああああああああああああああああああああああああああああああああああああああああああああああああああああああああああああああああああああああああああああああああああああああああああああああああああああああああああああああああああああああああああああああああああああああああああああああああああああああああああああああああああああああああああああああああああああああああああああああああああああああああああああああああああああああああああああああああああああああああああああああああああああああああああああああああああああああああああああああああああああああああああああああああ
\subsubsection{適用範囲の調整}
ここまで,耐戦略性を担保するために,チームでの活動にいかにしてグローヴスメカニズムを適用するかを考えてきた.

あああああああああああああああああああああああああああああああああああああああああああああああああああああああああああああああああああああああああああああああああああああああああああああああああああああああああああああああああああああああああああああああああああああああああああああああああああああああああああああああああああああああああああああああああああああああああああああああああああああああああああああああああああああああああああああああああああああああああああああああああああああああああああああああああああああああああああああああああああああああああああああああああああああああああああああああああああああああああああああああああああああああああああああああああああああああああああああああああああああああああああああああああああああああああああああああああああああああああああああああああああああああああああああああああああああああああ
\section{結論}
あああああああああああああああああああああああああああああああああああああああああああああああああああああああああああああああああああああああああああああああああああああああああああああああああああああああああああああああああああああああああああああああああああああああああああああああああああああああああああああああああああああああああああああああああああああああああああああああああああああああああああああああああああああああああああああああああああああああああああああああああああああああああああああああああああああああああああああああああああああああああああああああああああああああああああああああああああああああああああああああああああああああああああああああああああああああああああああああああああああああああああああああああああああああああああああああああああああああああああああああああああああああああああああああああああああああああ
\section{研究における課題と限界}
あああああああああああああああああああああああああああああああああああああああああああああああああああああああああああああああああああああああああああああああああああああああああああああああああああああああああああああああああああああああああああああああああああああああああああああああああああああああああああああああああああああああああああああああああああああああああああああああああああああああああああああああああああああああああああああああああああああああああああああああああああああああああああああああああああああああああああああああああああああああああああああああああああああああああああああああああああああああああああああああああああああああああああああああああああああああああああああああああああああああああああああああああああああああああああああああああああああああああああああああああああああああああああああああああああああああああ
\section*{参考文献}
\subsection*{引用文献}
\begingroup
\renewcommand{\section}[2]{}
\bibliography{references}
\bibliographystyle{junsrt}
\endgroup
\subsection*{参考文献}
\begin{itemize}[leftmargin=*]
\item[] 豊貴坂井. メカニズムデザイン : 資源配分制度の設計とインセンティブ / 坂井豊貴, 藤中裕二, 若山琢磨著. ミネルヴァ書房, 京都, 2008.8.
\end{itemize}
\end{document}
